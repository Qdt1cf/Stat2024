%\section{VA discrètes}
%Maîtriser le programme de CPGE.

%\section{VA absolument continues}

%\begin{itemize}
%	\item
%	Caractériser une variable aléatoire absolument continue, connaître la notion de densité de probabilité ;
%	\item
%	Fonction de répartition : connaître les définitions et propriétés ;
%	\item
%	Connaitre les lois usuelles et leurs propriétés : uniforme, exponentielle, Gamma, normale, 	$\chi^{2}$, Student ;
%	\item Centrer et réduire une variable aléatoire ;
%	\item
%	Calculer des probabilités à partir des tables de loi ;
%	\item
%	Calcul d'espérance par propriétés de calculs et le théorème de transfert ;
%	\item
%	Déterminer la loi d'un changement de variable (connaître différentes méthodes : fonction de répartition, théorème d'identification)
%	\item Identifier et construire un	mélange de lois. ;
%		\item
%	Fonction caractéristique d'une VA : connaître la définition, le lien avec la densité  et savoir les calculer ;
%\end{itemize}
%\section{Couples de VA}
%\begin{itemize}
%	\item 	Déterminer la loi d'un couple de variables aléatoires absolument continues ;

%	\item Déterminer les lois marginales d'un couple de VA
%	\item Caractériser l'indépendance de variables aléatoires et utiliser leurs propriétés ;
%	\item  Déterminer la loi d'une somme de variables aléatoires : méthode avec le produit de convolution, connaître les cas particuliers des lois  de Poisson et normales ;
%	\item Déterminer la	loi d'un changement de variable pour un couple de VA, méthode d'identification par changement de variables dans une intégrale double ;
%\end{itemize}

%\section{Théorèmes limites}
%\begin{itemize}

%	\item
%	Connaître les différents types de convergence : presque sûre, en probabilité, en loi ;
%	\item	Utiliser un théorème d'approximation d'une loi hypergéométrique par une loi binomiale
%	\item
%	Utiliser le théorème d'approximation d'une loi binomiale par une loi de Poisson ;
%	\item
%	Utiliser une approximation d'une somme de VA avec le Théorème Central Limite ;
%	\item	Calculer des probabilités en utilisant des approximations de lois.
%	\item Connaître la loi forte des grands nombres et savoir l'interpréter ;
%\end{itemize}
%\section{Méthodes de Monte Carlo}
%\begin{itemize}
%	\item
%	Effectuer un calcul approché d'intégrale par la méthode de Monte Carlo ;
%	\item
%	Réaliser un algorithme de calcul par une méthode de Monte Carlo ;
%	\item	Savoir contrôler l'erreur de la méthode
%\end{itemize}
%\section{Simulation de loi}

%\begin{itemize}
%	\item
%	Simuler une loi de probabilité par inversion de la fonction de répartition : calcul théorique et 	implémentation de quelques exemples ;
%	\item
%	Démontrer la validité et implémenter l'algorithme de Box-Müller ;
%	\item Interpréter une loi empirique (histogramme) ;
%\end{itemize}
\section{Échantillonnage}
\begin{itemize}
	\item
	Connaître les définitions de population, échantillon, réalisation d'un échantillon ;
%	\item Savoir distinguer un échantillonnage exhaustif et non exhaustif
	\item Estimateur : connaître les définitions et les propriétés (biais, convergence, efficacité) ;
	\item
	Appliquer la méthode du maximum de vraisemblance pour construire un estimateur pour des lois discrètes et continues ;
	\item
	Connaître les estimateurs usuels (moyenne, variance, proportion) et leurs propriétés ;
	\item
	Déterminer la loi exacte ou approchée des estimateurs usuels ;
	\item
	Construire un intervalle de confiance exact ou asymptotique  ;
\end{itemize}